%\documentclass[a4paper,twocolumn]{article} %两列
\documentclass[10pt,a4paper]{article}
\usepackage{titletoc} % 要调整章节标题在目录页中的格式,可以用titletoc宏包 title of contents
\usepackage{titlesec} % 其中 center 可使标题居中,还可设为 raggedleft (居左,默认),
\usepackage{abstract} % 摘要分栏的宏包
% \usepackage[OT1]{fontenc} % 添加后才使字体加粗和斜体起作用
\usepackage{xeCJK} % 中文字体
\usepackage{amsmath, amsthm}
\usepackage{listings,xcolor}
\usepackage{geometry} % 设置页边距
%\usepackage[margin=1in]{geometry}
%\usepackage{fontspec, xunicode, xltxtra}
\usepackage{fontspec}
\usepackage{graphicx} % 插图宏集
\usepackage{setspace}
\usepackage{fancyhdr} % 自定义页眉页脚
%\usepackage[bookmarks=true,colorlinks,linkcolor=black]{hyperref} %生成书签
\usepackage[breaklinks,colorlinks,linkcolor=black,citecolor=black,urlcolor=black]{hyperref}

\usepackage[Glenn]{fncychap}
\usepackage{color}
\usepackage{verbatim}
%\usepackage{tgpagella}

\definecolor{dkgreen}{rgb}{0,0.6,0}
\definecolor{gray}{rgb}{0.5,0.5,0.5}
\definecolor{mauve}{rgb}{0.58,0,0.82}
% 设置中文字体
\setCJKmainfont[BoldFont={华文楷体}, SlantedFont={华文楷体}, ItalicFont={方正公文楷体}]{华文楷体}
%\setCJKmainfont[BoldFont={华文楷体},SlantedFont={华文楷体},ItalicFont={方正行楷_GBK}]{华文楷体}
%\setCJKmainfont[BoldFont={华文楷体},SlantedFont={华文楷体},ItalicFont={楷体}]{华文楷体}
%\setCJKmainfont[BoldFont={华文楷体},SlantedFont={方正硬笔楷书_GBK},ItalicFont={方正硬笔行书简体}]{华文楷体}
\setCJKfamilyfont{华文楷体}{华文楷体}
%\setCJKfamilyfont{方正行楷_GBK}{方正行楷_GBK}
\setCJKmonofont{华文楷体}
%\setCJKmainfont[BoldFont={方正行楷_GBK},SlantedFont={方正行楷_GBK},ItalicFont={方正行楷_GBK}]{方正行楷_GBK}

% 设置英文字体
\setsansfont{Latin Modern Mono}
%\setsansfont{Consolas}
%\setsansfont{Sauce Code Pro Semibold Nerd Font Complete Mono}
\setmonofont[Mapping={}]{Latin Modern Mono} % 英文引号之类的正常显示,相当于设置英文字体
%\setmonofont[Mapping={}]{Consolas} % 英文引号之类的正常显示,相当于设置英文字体
%\setmonofont[Mapping={}]{Sauce Code Pro Semibold Nerd Font Complete Mono} % 英文引号之类的正常显示,相当于设置英文字体
\setmainfont{Latin Modern Mono}
%\setmainfont{Consolas}
%\setmainfont{Sauce Code Pro Semibold Nerd Font Complete Mono}
\linespread{1.2}

\title{Template For ICPC}
\author{ogmc}
\definecolor{dkgreen}{rgb}{0, 0.6, 0}
\definecolor{gray}{rgb}{0.5, 0.5, 0.5}
\definecolor{mauve}{rgb}{0.58, 0, 0.82}

\pagestyle{fancy}
% 以下分别为左中右的页眉和页脚
\CJKfamily{华文楷体}{
    \lhead{Nanchang University}
    \chead{}
    \rhead{第 \thepage 页}
    \lfoot{ogmc}
    \cfoot{\thepage}
    \rfoot{}
}
\renewcommand{\headrulewidth}{0.4pt}
\renewcommand{\footrulewidth}{0.4pt}
\renewcommand{\contentsname}{\huge \center 目录} % 目录标题
\geometry{left=2.5cm,right=3cm,top=2.5cm,bottom=2.5cm} % 页边距
\geometry{left=3.18cm,right=3.18cm,top=2.54cm,bottom=2.54cm}
\setlength{\columnsep}{30pt}

\makeatletter
\makeatother

\lstset{
    language = C++,
    numbers = left,
    numberstyle = {               % 设置行号格式
        \small
        \color{black}
        \fontspec{Consolas}
        },
    commentstyle ={               % 代码注释格式
        \color[RGB]{0,128,0}
        \bfseries
        },
    keywordstyle={                % 设置关键字格式
        \color[RGB]{40,40,255}
        \fontspec{Consolas}
        \bfseries
        },
    stringstyle={                 % 设置字符串格式
        %\color[RGB]{128,0,0}
        \color{red}
        \fontspec{Consolas}
        \bfseries
    },
    basicstyle={                  % 设置代码格式
        \fontspec{Consolas}
        \small
        },
    emphstyle   = \color[RGB]{112,64,160},   % 设置强调字格式
    breaklines  = true,                      % 设置自动换行
    tabsize     = 4,
    frame       = single,                    % 主题
    columns     = fullflexible,
    showstringspaces = false,                % 不显示代码字符串中间的空格标记
    aboveskip=3mm,
    belowskip=3mm,
    breakatwhitespace=true
    tabsize=3
}

\begin{document}
\begin{titlepage}
    \newcommand{\HRule}{\rule{\linewidth}{0.1mm}}
    \center
    \quad\\[1.5cm]
    %\textsl{\Large University of Chinese Academy of Sciences }\\[0.5cm]
    %\textsl{\large School of Electrical and Electronic Engineering}\\[0.5cm]
    \makeatletter
    \HRule \\[0.4cm]
    \emph{\fontsize{40pt}{\baselineskip}\selectfont \bfseries \@title}\\[0.2cm]
    \HRule \\[5cm]
    \includegraphics[width=10cm]{logo.jpeg}\\[1cm]
    \vskip 5cm
    \begin{minipage}{1\textwidth}
        \begin{center} \LARGE
            \emph{Author: \@author} \\
            \emph{Email: haoran.mc@outlook.com} \\
        \end{center}
    \end{minipage}
    \makeatother
    \vfill
    \thispagestyle{empty}
    \pagebreak
    \pagestyle{plain}
    \tableofcontents
\end{titlepage}
\section{01-基本算法}
\subsection{01-lower-upper-bound}
\lstinputlisting{../01-基本算法/01-lower-upper-bound.cpp}
\subsection{01-三次方根}
\lstinputlisting{../01-基本算法/01-三次方根.cpp}
\subsection{01-根据厚度将书分组}
\lstinputlisting{../01-基本算法/01-根据厚度将书分组.cpp}
\subsection{01-离散化-区间和}
\lstinputlisting{../01-基本算法/01-离散化-区间和.cpp}
\subsection{01-简单递归-for循环}
\lstinputlisting{../01-基本算法/01-简单递归-for循环.cpp}
\subsection{01-糖果传递}
\lstinputlisting{../01-基本算法/01-糖果传递.cpp}
\subsection{01-递归+vector}
\lstinputlisting{../01-基本算法/01-递归+vector.cpp}
\subsection{02-lhr}
\lstinputlisting{../01-基本算法/02-lhr.cpp}
\subsection{02-swap}
\lstinputlisting{../01-基本算法/02-swap.cpp}
\subsection{02-七夕祭}
\lstinputlisting{../01-基本算法/02-七夕祭.cpp}
\subsection{02-递归+哈希}
\lstinputlisting{../01-基本算法/02-递归+哈希.cpp}
\subsection{02-递归+状压}
\lstinputlisting{../01-基本算法/02-递归+状压.cpp}
\subsection{03-递归+状压}
\lstinputlisting{../01-基本算法/03-递归+状压.cpp}
\subsection{03-递归全排列+状压}
\lstinputlisting{../01-基本算法/03-递归全排列+状压.cpp}
\subsection{04-quickSort}
\lstinputlisting{../01-基本算法/04-quickSort.cpp}
\subsection{05-第k个数}
\lstinputlisting{../01-基本算法/05-第k个数.cpp}
\subsection{06-mergeSort}
\lstinputlisting{../01-基本算法/06-mergeSort.cpp}
\subsection{07-逆序数}
\lstinputlisting{../01-基本算法/07-逆序数.cpp}
\subsection{一维前缀和}
\lstinputlisting{../01-基本算法/一维前缀和.cpp}
\subsection{一维差分}
\lstinputlisting{../01-基本算法/一维差分.cpp}
\subsection{二维前缀和}
\lstinputlisting{../01-基本算法/二维前缀和.cpp}
\subsection{二维差分}
\lstinputlisting{../01-基本算法/二维差分.cpp}
\subsection{均分纸牌}
\lstinputlisting{../01-基本算法/均分纸牌.cpp}
\section{02-搜索}
\subsection{01-树的深度优先搜索}
\lstinputlisting{../02-搜索/01-树的深度优先搜索.cpp}
\subsection{02-图的深度优先搜索}
\lstinputlisting{../02-搜索/02-图的深度优先搜索.cpp}
\subsection{02-木棍dfs剪枝}
\lstinputlisting{../02-搜索/02-木棍dfs剪枝.cpp}
\subsection{03-树的广度优先搜索}
\lstinputlisting{../02-搜索/03-树的广度优先搜索.cpp}
\subsection{04-图的广度优先搜索}
\lstinputlisting{../02-搜索/04-图的广度优先搜索.cpp}
\subsection{BFS+Cantor}
\lstinputlisting{../02-搜索/BFS+Cantor.cpp}
\subsection{子集与二进制关系}
\lstinputlisting{../02-搜索/子集与二进制关系.cpp}
\subsection{组合与二进制关系}
\lstinputlisting{../02-搜索/组合与二进制关系.cpp}
\section{03-动态规划}
\subsection{01-Bone-Collector}
\lstinputlisting{../03-动态规划/01-Bone-Collector.cpp}
\subsection{01-Corn-Fields}
\lstinputlisting{../03-动态规划/01-Corn-Fields.cpp}
\subsection{01-hdu1257}
\lstinputlisting{../03-动态规划/01-hdu1257.cpp}
\subsection{01-二维费用的背包问题}
\lstinputlisting{../03-动态规划/01-二维费用的背包问题.cpp}
\subsection{01-多重背包}
\lstinputlisting{../03-动态规划/01-多重背包.cpp}
\subsection{01-完全背包问题}
\lstinputlisting{../03-动态规划/01-完全背包问题.cpp}
\subsection{1-拦截子弹++}
\lstinputlisting{../03-动态规划/1-拦截子弹++.cpp}
\subsection{01-最少硬币问题}
\lstinputlisting{../03-动态规划/01-最少硬币问题.cpp}
\subsection{01-最长公共子序列}
\lstinputlisting{../03-动态规划/01-最长公共子序列.cpp}
\subsection{1-木棍加工++}
\lstinputlisting{../03-动态规划/1-木棍加工++.cpp}
\subsection{01-混合背包}
\lstinputlisting{../03-动态规划/01-混合背包.cpp}
\subsection{01-石子合并}
\lstinputlisting{../03-动态规划/01-石子合并.cpp}
\subsection{02-一维}
\lstinputlisting{../03-动态规划/02-一维.cpp}
\subsection{02-一维1}
\lstinputlisting{../03-动态规划/02-一维1.cpp}
\subsection{02-不要4-记忆化}
\lstinputlisting{../03-动态规划/02-不要4-记忆化.cpp}
\subsection{02-二进制分组优化}
\lstinputlisting{../03-动态规划/02-二进制分组优化.cpp}
\subsection{02-回文串}
\lstinputlisting{../03-动态规划/02-回文串.cpp}
\subsection{2-拦截导弹}
\lstinputlisting{../03-动态规划/2-拦截导弹.cpp}
\subsection{02-最少硬币组合}
\lstinputlisting{../03-动态规划/02-最少硬币组合.cpp}
\subsection{03-单调队列优化}
\lstinputlisting{../03-动态规划/03-单调队列优化.cpp}
\subsection{3-导弹拦截}
\lstinputlisting{../03-动态规划/3-导弹拦截.cpp}
\subsection{03-所有硬币组合-1}
\lstinputlisting{../03-动态规划/03-所有硬币组合-1.cpp}
\subsection{04-所有硬币组合-2}
\lstinputlisting{../03-动态规划/04-所有硬币组合-2.cpp}
\section{04-字符串}
\subsection{01-KMP}
\lstinputlisting{../04-字符串/01-KMP.cpp}
\subsection{02-Manacher}
\lstinputlisting{../04-字符串/02-Manacher.cpp}
\subsection{04-最长公共子串}
\lstinputlisting{../04-字符串/04-最长公共子串.cpp}
\section{05-数学问题}
\subsection{01-Bash-Game-sg}
\lstinputlisting{../05-数学问题/01-Bash-Game-sg.cpp}
\subsection{01-Catalan}
\lstinputlisting{../05-数学问题/01-Catalan.cpp}
\subsection{01-gcd-lcm}
\lstinputlisting{../05-数学问题/01-gcd-lcm.cpp}
\subsection{01-快速幂测试}
\lstinputlisting{../05-数学问题/01-快速幂测试.cpp}
\subsection{01-排列组合}
\lstinputlisting{../05-数学问题/01-排列组合.cpp}
\subsection{01-整数除以2}
\lstinputlisting{../05-数学问题/01-整数除以2.cpp}
\subsection{01-素数}
\lstinputlisting{../05-数学问题/01-素数.cpp}
\subsection{01-费马小定理}
\lstinputlisting{../05-数学问题/01-费马小定理.cpp}
\subsection{01-进制转换}
\lstinputlisting{../05-数学问题/01-进制转换.cpp}
\subsection{02-二进制状态压缩}
\lstinputlisting{../05-数学问题/02-二进制状态压缩.cpp}
\subsection{02-埃氏筛法}
\lstinputlisting{../05-数学问题/02-埃氏筛法.cpp}
\subsection{02-扩展欧几里得}
\lstinputlisting{../05-数学问题/02-扩展欧几里得.cpp}
\subsection{02-扩展欧几里得算法}
\lstinputlisting{../05-数学问题/02-扩展欧几里得算法.cpp}
\subsection{03-lowbit}
\lstinputlisting{../05-数学问题/03-lowbit.cpp}
\subsection{03-wythoff-Game}
\lstinputlisting{../05-数学问题/03-wythoff-Game.cpp}
\subsection{03-快速幂模板}
\lstinputlisting{../05-数学问题/03-快速幂模板.cpp}
\subsection{03-欧拉筛法}
\lstinputlisting{../05-数学问题/03-欧拉筛法.cpp}
\subsection{03-递推}
\lstinputlisting{../05-数学问题/03-递推.cpp}
\subsection{04-n!中质因子p的个数}
\lstinputlisting{../05-数学问题/04-n!中质因子p的个数.cpp}
\subsection{04-终极递推}
\lstinputlisting{../05-数学问题/04-终极递推.cpp}
\subsection{04-高精度加法}
\lstinputlisting{../05-数学问题/04-高精度加法.cpp}
\subsection{05-素因子}
\lstinputlisting{../05-数学问题/05-素因子.cpp}
\subsection{05-高精度减法}
\lstinputlisting{../05-数学问题/05-高精度减法.cpp}
\subsection{06-质因子分解}
\lstinputlisting{../05-数学问题/06-质因子分解.cpp}
\subsection{06-高精度乘法}
\lstinputlisting{../05-数学问题/06-高精度乘法.cpp}
\subsection{07-高精度除法}
\lstinputlisting{../05-数学问题/07-高精度除法.cpp}
\subsection{Fibonacci}
\lstinputlisting{../05-数学问题/Fibonacci.cpp}
\section{06-数据结构}
\subsection{01-Segment-Tree}
\lstinputlisting{../06-数据结构/01-Segment-Tree.cpp}
\subsection{01-Sliding-Window}
\lstinputlisting{../06-数据结构/01-Sliding-Window.cpp}
\subsection{01-单调栈}
\lstinputlisting{../06-数据结构/01-单调栈.cpp}
\subsection{02-最大子序列和}
\lstinputlisting{../06-数据结构/02-最大子序列和.cpp}
\subsection{02-逆序链表迭代}
\lstinputlisting{../06-数据结构/02-逆序链表迭代.cpp}
\subsection{Template-并查集}
\lstinputlisting{../06-数据结构/Template-并查集.cpp}
\subsection{对顶堆}
\lstinputlisting{../06-数据结构/对顶堆.cpp}
\subsection{并查集合并压缩}
\lstinputlisting{../06-数据结构/并查集合并压缩.cpp}
\subsection{并查集路径压缩}
\lstinputlisting{../06-数据结构/并查集路径压缩.cpp}
\subsection{数状数组}
\lstinputlisting{../06-数据结构/数状数组.cpp}
\section{07-图论}
\subsection{01-BFS-邻接矩阵}
\lstinputlisting{../07-图论/01-BFS-邻接矩阵.cpp}
\subsection{01-DFS-邻接矩阵}
\lstinputlisting{../07-图论/01-DFS-邻接矩阵.cpp}
\subsection{01-Floyd}
\lstinputlisting{../07-图论/01-Floyd.cpp}
\subsection{01-kruskal}
\lstinputlisting{../07-图论/01-kruskal.cpp}
\subsection{01-prim}
\lstinputlisting{../07-图论/01-prim.cpp}
\subsection{01-SPFA}
\lstinputlisting{../07-图论/01-SPFA.cpp}
\subsection{01-邻接矩阵}
\lstinputlisting{../07-图论/01-邻接矩阵.cpp}
\subsection{02-BFS-邻接表}
\lstinputlisting{../07-图论/02-BFS-邻接表.cpp}
\subsection{02-DFS-邻接表}
\lstinputlisting{../07-图论/02-DFS-邻接表.cpp}
\subsection{02-二叉堆}
\lstinputlisting{../07-图论/02-二叉堆.cpp}
\subsection{02-拓扑排序}
\lstinputlisting{../07-图论/02-拓扑排序.cpp}
\subsection{03-BFS-链式前向星}
\lstinputlisting{../07-图论/03-BFS-链式前向星.cpp}
\subsection{03-DFS-链式前向星}
\lstinputlisting{../07-图论/03-DFS-链式前向星.cpp}
\subsection{Bellman-Ford}
\lstinputlisting{../07-图论/Bellman-Ford.cpp}
\section{08-杂项}
\subsection{mt19937}
\lstinputlisting{../08-杂项/mt19937.cpp}
\subsection{shuffle}
\lstinputlisting{../08-杂项/shuffle.cpp}
\subsection{约瑟夫环}
\lstinputlisting{../08-杂项/约瑟夫环.cpp}
\end{document}
